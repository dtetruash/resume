% !TeX root = ./resume.tex
%!TEX TS-program = xelatex
%!TEX encoding = UTF-8 Unicode
% Awesome CV LaTeX Template for CV/Resume
%
% This template has been downloaded from:
% https://github.com/posquit0/Awesome-CV
%
% Author:
% Claud D. Park <posquit0.bj@gmail.com>
% http://www.posquit0.com
%
% Template license:
% CC BY-SA 4.0 (https://creativecommons.org/licenses/by-sa/4.0/)
%


%-------------------------------------------------------------------------------
% CONFIGURATIONS
%-------------------------------------------------------------------------------
% A4 paper size by default, use 'letterpaper' for US letter
\documentclass[11pt, a4paper]{awesome-cv}

% Configure page margins with geometry
\geometry{left=1.4cm, top=.8cm, right=1.4cm, bottom=1.8cm, footskip=.5cm}

% Specify the location of the included fonts
\fontdir[fonts/]

% Color for highlights
% Awesome Colors: awesome-emerald, awesome-skyblue, awesome-red, awesome-pink, awesome-orange
%                 awesome-nephritis, awesome-concrete, awesome-darknight
\colorlet{awesome}{awesome-orange}
% Uncomment if you would like to specify your own color
% \definecolor{awesome}{HTML}{CA63A8}

% Colors for text
% Uncomment if you would like to specify your own color
\definecolor{darktext}{HTML}{414141}
\definecolor{text}{HTML}{3D3D3D}
\definecolor{graytext}{HTML}{3D3D3D}
\definecolor{lighttext}{HTML}{3D3D3D}

% Set false if you don't want to highlight section with awesome color
\setbool{acvSectionColorHighlight}{false}

% If you would like to change the social information separator from a pipe (|) to something else
\renewcommand{\acvHeaderSocialSep}{\quad\textbar\quad}

%-------------------------------------------------------------------------------
% better undeline command by alexwlchan (https://alexwlchan.net/2017/10/latex-underlines/)
\usepackage{contour}
\usepackage{ulem}

\renewcommand{\ULdepth}{1.8pt}
\contourlength{0.8pt}

\newcommand{\underl}[1]{%
  \uline{\phantom{#1}}%
  \llap{\contour{white}{#1}}%
}

% Bullet Seperator
\newcommand{\bsep}{~~•~~}
%-------------------------------------------------------------------------------

%-------------------------------------------------------------------------------
%	PERSONAL INFORMATION
%	Comment any of the lines below if they are not required
%-------------------------------------------------------------------------------
% Available options: circle|rectangle,edge/noedge,left/right
\photo[rectangle,edge,right]{profile}
\name{David Simon}{Tetruashvili}
\position{Postgraduate Computer Science Student}
\address{Zurich, Switzerland.\bsep Berlin, Germany.}

\mobile{+41 (0) 76-723-3936}
\email{david.tetruashvili@proton.me}
%\homepage{}
\github{davzzar}
%\linkedin{DavidSTetruashvli}
% \gitlab{gitlab-id}
% \stackoverflow{SO-id}{SO-name}
% \twitter{@twit}
% \skype{skype-id}
% \reddit{reddit-id}
% \medium{madium-id}
% \googlescholar{googlescholar-id}{name-to-display}
%% \firstname and \lastname will be used
% \googlescholar{googlescholar-id}{}
% \extrainfo{extra informations}

%\quote{``The best opportunity comes when you find you need something that does not yet exist.''}


\begin{document}

% Print the header with above personal informations
% Give optional argument to change alignment(C: center, L: left, R: right)
\makecvheader[C]

% Print the footer with 3 arguments(<left>, <center>, <right>)
% Leave any of these blank if they are not needed
\makecvfooter
  {\today}
    {David S. Tetruashvli\bsep Curriculum Vitae}
  {\thepage}



%-------------------------------------------------------------------------------
%	CV/RESUME CONTENT
%	Each section is imported separately, open each file in turn to modify content
%-------------------------------------------------------------------------------
%%-------------------------------------------------------------------------------
%	SECTION TITLE
%-------------------------------------------------------------------------------
\cvsection{Profile}


%-------------------------------------------------------------------------------
%	CONTENT
%-------------------------------------------------------------------------------
\begin{cvparagraph}

%---------------------------------------------------------
% I am a striving trilingual ETH Computer Science Masters student with an international background. I am keen on deepening myself in the environment that ETH AI Center offers. Being a science communicator though my experiences of various teaching positions, I can deliver concepts, ideas, and information concisely. With skills in graphic design and video production, I hope to engage the AI Center’s audience with its outreach goals. Having organized several large celebratory and sports events, as well as small personal/group projects, I believe I am suitable in the role of a Student Assistant.

Having just finished my Autumn Semester examination session I am keen to apply myself in a semester project based in machine learning, computer vision and graphics, or (preferably) the intersection thereof. With the experience from previous courses as Physically-based Simulation, Computer Vision, and Advanced Machine Learning fresh in my working memory, as well as project other project experiences both at ETH and elsewhere, I look forward to work with you on our shared area of interest in this semester project.
\end{cvparagraph}

%-------------------------------------------------------------------------------
%	SECTION TITLE
%-------------------------------------------------------------------------------
\cvsection{Education}


%-------------------------------------------------------------------------------
%	CONTENT
%-------------------------------------------------------------------------------
\begin{cventries}

    \cventry
    {Master of Science in Computer Science}
    {Swiss Federal Institute of Technology (ETH Zurich)}
    {Zurich, Switzerland}
    {Sept. 2020 --- Present}
    {
        \begin{cvitems}
            \item Relevant courses:
            \begin{flushleft}
                \underl{Game Programming Lab (C\#/.NET)}
                \bsep \underl{Advanced Machine Learning}
                \bsep \underl{Computer Vision}
                \bsep \underl{3D Vision} \\
                \underl{Physically-based Simulation in Computer Graphics (C++)}
                %\bsep \underl{Mathematical Foundations of Computer Graphics and Vision} \bsep
                \bsep \underl{Machine Perception}
                %\underl{Natural Language Processing}
                \end{flushleft}
        \end{cvitems}
    }

    %---------------------------------------------------------
    \cventry
    {Bachelor of Science in Computer Science (with Artificial Intelligence) (Hons.)} % Degree
    {King’s College London --- University of London} % Institution
    {London, UK} % Location
    {Sep. 2017 --- Aug. 2020} % Date(s)
    {
        \begin{cvitems} % Description(s) bullet points
            \item {\textbf{First-class degree. (GPA: 86\%)}}
            \item {Relevant modules include:
                        \begin{flushleft}
                            %\underl{Cryptography}, \\
                            \underl{Information Security}
                            %\underl{Optimization Methods} \bsep
                            %\underl{Machine Learning} \\
                            \bsep \underl{Internet Systems}  \textit{[\textbf{96\% (Top mark in class)}]}
                            \bsep \underl{Database Systems (MySQL)} \textit{[\textbf{90\%}]}\\
                            \underl{Software Engineering Group Project} \textit{[Major Project: \textbf{91\%}, Written Examination: \textbf{92\% (Top mark in class)]}}\\
                            %\underl{Artificial Intelligence Reasoning and Decision Making} \textit{[Coursework: \textbf{95\%}, Overall: \textbf{86\%}]} \bsep
                            %\underl{Artificial Intelligence Planning} \textit{[\textbf{82\%}]} \\
                            %\underl{Introduction to Artificial Intelligence} \textit{[Coursework \#1: \textbf{85\%}, Coursework \#2: \textbf{90\%}, Overall: \textbf{84\%}]} \\
                            \underl{Programming Practice and Applications (Java)} \textit{[\textbf{96\%}]},
                            \underl{Practical Experiences of Programming (C++/Scala)} \textit{[\textbf{91\%}]}, \\
                            %\underl{Elementary Logic With Applications} \textit{[\textbf{92\%}]},
                            %\underl{Computer Systems} \textit{[\textbf{90\%}]},
                            \underl{Introduction to Software Engineering} \textit{[\textbf{90\%}]},
                            %\underl{Foundations of Computing II} \textit{[\textbf{88\%}]}.
                            %\underl{Data Structures} \textit{[\textbf{87\%}]}
                            %\underl{Programming Language Design Paradigms} \textit{[\textbf{82\%}]}.
                            \underl{Operating Systems and Concurrency} \textit{[\textbf{73\%}]},\\
                        \end{flushleft}
                  }
        \end{cvitems}
1e    }

    %---------------------------------------------------------
    \cventry
    {International Baccalaureate Diploma} % Degree
    {Berlin International School --- Private Kant-Schule} % Institution
    {Berlin, Germany} % Location
    {Sep. 2015 --- May 2017} % Date(s)
    {
        \begin{cvitems} % Description(s) bullet points
            \item {\textbf{Overall 42/45 points}, including \textbf{6/7 points} in Higher Level Mathematics.
            %\emph{\textbf{7/7 points} in Higher Level Film} by the end of which I produced, directed, and edited a 7-minute short film.
            }
            %\item {Extended Essay in Computer Science on \textit{Evaluation of Texture Filtering Algorithms applied in VR HUDs}:\\
            %\textbf{33/36 points, Grade A}.}
        \end{cvitems}
    }

    %---------------------------------------------------------
    \cventry
    {International-GCSE} % Degree
    {} % Institution
    {} % Location
    {Sep. 2013 --- May 2015} % Date(s)
    {
        \begin{cvitems} % Description(s) bullet points
            \item {10 IGCSEs with eight being A* or A, including A*s in Mathematics, Physics, and ICT.}
        \end{cvitems}
    }

    %---------------------------------------------------------
\end{cventries}

%-------------------------------------------------------------------------------
%	SECTION TITLE
%-------------------------------------------------------------------------------
\cvsection{Experience}


%-------------------------------------------------------------------------------
%	CONTENT
%-------------------------------------------------------------------------------
\begin{cventries}

% \cventry
%   {King's College London}
%   {Teaching Assistant}
%   {London, UK}
%   {Jan. 2020 --- Jan. 2021}
%   {
%     \begin{cvitems}
%       \bsep{} {Lead tutorial sessions for groups of 40+ students in Programming Language Design, Introductory Software Engineering, and Operating Systems and Cuncurrency.}
%       \bsep{} {Prepared and give short and hands-on lectures on relevant compsci-theretic topics (e.g., structural induction on trees).} \bsep{} Built rapport with students (both in-person and remotely) for a successful teaching experience.
%     \end{cvitems}
%   }
  
%---------------------------------------------------------
%   \cventry
%   {TeachFirst (Placed at Harris Academy Peckham)} % Organisation
%   {Insight Programme Intern} % Job Title
%   {} % Location
%   {Jun. 2019} % Date(s)
%   {
%     \begin{cvitems} % Description(s) of tasks/responsibilities
%       \bsep{} Underwent crash-course-style pedagogic training focusing on self-development in problem solving, leadership, and innovation through workshops, lectures, and practical exercises.
%       \bsep{} Spent one week at Harris Academy Peckham serving as teaching assistant in the department of mathematics.
%       \bsep{} Planned, prepared resources for, and delivered a mathematics lesson to a year 8 set 1 class.
%       \bsep{} Observing teacher noted \textit{``outstanding use of pre-prepared resources, great rapport built with students, and positive and motivational attitude.''}
%     \end{cvitems}
%   }

%---------------------------------------------------------
  \cventry
    {Guy's and St.\ Thomas’ Hospital NHS Trust} % Organization
    {Mobile App Developer \& Database Engineer} % Job title
    {London, UK} % Location
    {Feb. --- Mar. 2019} % Date(s)
    {
      \begin{cvitems} % Description(s) of tasks/responsibilities
        \item {In a team of eight, I developed a Doctor-Patient communication and medical test compliance \emph{cross-platform mobile app} called \href{https://dtetruash.github.io/prep-page/}{\textbf{\uline{``Prep.''}}}}
        \item I designed and implemented the app's UI with focus on accessibility, utilizing Google’s Flutter framework.
        \bsep{} {I designed and implemented the back-end system with emphasis on privacy, data leak prevention, and scalability using Google’s NoSQL Firebase}
        \bsep{} I managed and maintained the project's Git history
        \item {I created and narrated a \href{https://drive.google.com/file/d/1Pu6NSpqPXnWcaanUpIu_gd5C1bQnOyQi/preview}{\underl{screencast}} presenting the final application.}
        \item {Our solution has been shortlisted for funding.}
      \end{cvitems}
    }

%---------------------------------------------------------
\end{cventries}

%-------------------------------------------------------------------------------
%	SECTION TITLE
%-------------------------------------------------------------------------------
\cvsection{Projects}


%-------------------------------------------------------------------------------
%	CONTENT
%-------------------------------------------------------------------------------
\begin{cventries}

%---------------------------------------------------------
    \cventry
    {Game Developer/Designer, Git Master, Asset Creation}
    {\href{https://www.youtube.com/watch?v=u3z3P-rHndg}{\uline{Seth's PyraMall}} --- A Competitive Couch Platformer}
    {Zurich, Switzerland}
    {Feb. 2022 -- May. 2022}
    {
        \begin{cvitems}
        \item As a part of the \href{https://gtc.inf.ethz.ch/education/game-programming-laboratory.html}{\uline{Game Technology Center}}'s Game Programming Lab, in a team of six, I developed a 2D retro-style platformer game using C\#, .NET, and the Monogame framework.
        \item Led the team: set priorities, defined  and assigned individual tasks,  mainated theproject's issue log, as well as the projects Git etiquette and history.
        \item As a developer, I developed the character controller, contributed to the sound-effect system and the menu GUI system.
        \item I learned how to apply C\# and .NET during the project. How to apply OOP design-patterns in the language, as well as how to keep a clean and well-documented codebase.
        \item I \href{https://youtu.be/AuFtp9p35H4?t=1211}{presented the final game to an audience of c. 200}; won the \href{https://youtu.be/AuFtp9p35H4?t=5614}{\textbf{\uline{Jury Award}} from the triple-A Studio Gobo} for the game.
        \end{cvitems}
    }

    %---------------------------------------------------------
    \cventry
    {Developer}
    {Tool for Analyzing Projects and Resouce Allocation Trends}
    {Zurich, Switzerland}
    {Sep. 2021 -- Jan. 2022}
    {
        \begin{cvitems}
        \item As a part of the \href{https://analytics-club.org/wordpress/hack4good/}{\uline{Hack4Good}} program, in a team of four, I developed an NLP analysis package for \href{https://www.helvetas.org/en/switzerland}{\uline{Helvetas}} to help the company better understand the distribution of their decentrelized internaltional projects.
        \item I developed and tested NLP models for the analysis, implemented the package's business logic and partly designed its web-based UI. I used Docker to share the prototype with the company.
        \item The package allows the user to analyze a dataset of projects, classify them into areas of development, and identify trends appearing in their descriptions and outcomes to better allocate resources.
        \end{cvitems}
    }

    %---------------------------------------------------------
    % \cventry
    % {Developer}
    % {Classification of Heart Arrhythmia from ECG signals using DL}
    % {Zurich, Switzerland}
    % {Nov. 2021}
    % {
    %     \begin{cvitems}
    %     \item I implemented and adapted the Stanford ECG2 architecture using PyTorch for submission during Advanced Machine Learning at ETH Zurich.
    %     \item Deployed remote GPUs for accelerated training using Docker images.
    %     \item Solution improved team's leaderboard position from c. 140th to 6th place out of c. 200 teams.
    %     \end{cvitems}
    % }
    
    % %---------------------------------------------------------
    % \cventry
    % {Developer}
    % {Analyzing and Augmenting PWC-Net for Multi-Human Optical Flow}
    % {Zurich, Switzerland}
    % {May. 2020}
    % {
    %     \begin{cvitems}
    %     \item In a team of three, I analyzed and carried out augmentations to the PyTorch implementation of PWC-Net for Human Optical Flow as the final project for Final Project in Machine Perception at ETH Zurich.
    %     \item I attempted to improve the network's overfitting resilience via convolutional DropBlock and a revised data augmentation pipeline.
    %     \item I designed and carried out iterative experiments on remote GPUs to train and tune the model.
    %     \end{cvitems}
    % }

    %---------------------------------------------------------
    \cventry
    {Graphic and UI/UX Designer, Frontend Developer} % Job title
    {\href{https://github.com/davzzar/clim-EX}{\uline{Clim-EX}}: The Climate Animation Explorer} % Organization
    {Oxford, UK} % Location
    {Nov. 2019} % Date(s)
    {
        \begin{cvitems}
        \item Selected to participate \href{http://www.ox.ac.uk/students/news/2019-10-14-oxford-hack-2019}{\uline{Oxford University Hackathon 2019}. In a team of four, built a project promoting climate change discussion and exploration.
        \item Designed and built the frontend. Our submission was one of the lead contenders to the ``Hacker's Choice'' award at the hackathon.
        \end{cvitems}
    }

    %---------------------------------------------------------
    \cventry
    {Designer and Programmer} % Job title
    {\href{https://github.com/dtetruash/solved-af}{\uline{Solved-AF}}: Argumentation Framework Solver and ``SAF-link'' API} % Organization
    {London, UK} % Location
    {Oct. 2019 --- Aug. 2020} % Date(s)
    {
        \begin{cvitems}
            \item I designed and developed an \href{https://www.sciencedirect.com/science/article/pii/000437029400041X}{\underl{Argumentation Framework}} solver for non-monotonic reasoning as part of my Bachelor's Thesis with the goal of producing a solver for educational and exploratory purposes.
            \item To aid in the exploratory role, I developed a REST API prototype which enables web clients to use the rich the plethora of existing Argumentation Framework solvers bypassing the knowledge barrier to use. 
            \item The thesis achieved a first-class mark, in part due to the implementation.
        \end{cvitems}
    }

    %---------------------------------------------------------
    % \cventry
    % {Designer and Programmer} % Job title
    % {Collegiate Information Portal (KCL-info)} % Organization
    % {} % Location
    % {Oct. 2019 --- Present} % Date(s)
    % {
    %     \begin{cvitems}
    %         \bsep Currently developing an information retrieval web API for use by staff and students of KCL. The main purpose is to provide easy access to important and up-to-date information around campus and all of its aspects in one unified place.
    %     \end{cvitems}
    % }
\end{cventries}

%---------------------------------------------------------

%-------------------------------------------------------------------------------
%	SECTION TITLE
%-------------------------------------------------------------------------------
\cvsection{Skills}


%-------------------------------------------------------------------------------
%	CONTENT
%-------------------------------------------------------------------------------
\begin{cvskills}

%---------------------------------------------------------
\cvskill
  {Languages} % Category
  {English, German, Russian.} % Skills
%---------------------------------------------------------
%   \cvskill
%     {Image Production} % Category
%     {Image Editing (Adobe Photoshop, GIMP), Vector Graphics (Gravit Designer/Ink Scape).} % Skills
    
%---------------------------------------------------------
%   \cvskill
%     {Video Production} % Category
%     {Video Editing (Adobe Premiere Pro, Apple FinalCut X), Filming/Photography Equipment.} % Skills
    

%---------------------------------------------------------
  \cvskill
    {Programming} % Category
    {Python, C/C++, C\#, Go, Java.} % Skills
    
    \cvskill
    {Machine Learning} % Category
    {PyTorch, NumPy, Pandas, SciPy, Matplotlib, etc.} % Skills

%---------------------------------------------------------
  \cvskill
    {Systems} % Category
    {Linux, Docker, SSH/Remote Access, Zsh/Bash scripting \& automation, Git \& GitHub/GitLab.} % Skills

%---------------------------------------------------------
  \cvskill
    {Tools}
    {(Microsoft) Office Suite, IaaS (Digital Ocean, Vast.ai), LaTeX.}
\end{cvskills}

%-------------------------------------------------------------------------------
%	SECTION TITLE
%-------------------------------------------------------------------------------
\cvsection{Awards}
\begin{cvhonors}
%---------------------------------------------------------
  \cvhonor
    {Studio Gobo's Jury Award} % Award
    {Game Programming Labaratory, ETH Zurich} % Event
    {Zurich, Switzerland} % Location
    {2022} % Date(s)

%---------------------------------------------------------
  \cvhonor
    {Nomination} % Award
    {Informatics Outstanding Teaching Award, KCL Informatics Department} % Event
    {London, UK} % Location
    {2020} % Date(s)

\end{cvhonors}
%-------------------------------------------------------------------------------
%	SECTION TITLE
%-------------------------------------------------------------------------------
\cvsection{Extracurricular Activity \& Volunteering}


%-------------------------------------------------------------------------------
%	CONTENT
%-------------------------------------------------------------------------------
\begin{cventries}

    %---------------------------------------------------------
  \cventry
  {Makerspace Manager} % Job Title
  {ETH Zurich Student Project House} % Organisation
  {Zurich, Switzerland} % Location
  {April. 2022 --- Present} % Date(s)
  {
    \begin{cvitems} % Description(s) of tasks/responsibilities
      \item I volunteere as a makerspace manager at the Student Project House at ETH Zurch.
      \item My responsibilities include giving introductory lectures/workshop on the in-house machinery and equipment, giving general introductions to the makerspace, supervising the makerspace and manning the counter during open-hours.
    \end{cvitems}
  }
  
  %---------------------------------------------------------
  \cventry
  {Member} % Job Title
  {KCL Informatics and Engineering Teaching Assistant Liaison Committee} % Organisation
  {London, UK} % Location
  {Jan. --- Aug. 2020} % Date(s)
  {
    \begin{cvitems} % Description(s) of tasks/responsibilities
      \item I acted as a contact point to the TA community in case of concerns with module leaders and TA-related procedures.
    \end{cvitems}
  }

  \cventry
  {Temporary Member/Active Participant} % Job Title
  {KCL Informatics and Engineering Student-Staff Liaison Committee} % Organisation
  {} % Location
  {Dec. 2019} % Date(s)
  {
    \begin{cvitems} % Description(s) of tasks/responsibilities
      \item {I volunteered time to raise and discuss possible solutions to student concerns about the KCL `Artificial Intelligence Planning' module with both the administration of the faculty and the module leaders Dr Stefan Edelkamp \& Dr Daniele Magazzeni as a member of the student body.}
    \end{cvitems}
  }

%  ---------------------------------------------------------
  \cventry
  {German and Russian Language Tutor} % Job Title
  {Modern Language Center at King’s College London} % Organisation
  {} % Location
  {Apr. --- Aug. 2019} % Date(s)
  {
     \begin{cvitems} % Description(s) of tasks/responsibilities
    \item {I volunteered time to run speaking practice sessions with students at the university and persons of the public.}
    \item {I evaluated and gave feedback on the students' use of the language and lexicon ultimately to help them express themselves in the language.}
    \item {I helped students build confidence in speaking the language in a safe environment.}
    \end{cvitems}
 }

%  ---------------------------------------------------------
  \cventry
  {Workshop Leader and Organizer} % Job Title
  {Beginner's Programming and Game Development Workshop} % Organisation
  {Berlin, Germany} % Location
  {May. --- Jul. 2017} % Date(s)
  {
    \begin{cvitems} % Description(s) of tasks/responsibilities
      \item I founded and launched an after-school beginner’s programming workshop after learning that the school did not offer adequate education in the topic `due to low demand.'
      \item I thought the basics of programming to a class of c. 30 students of mixed ages using Python 3 and the 2D game engine Game Maker Studio.
      \item This effort was acknowledged by the school and the administration, and was noted in that year's yearbook.
    \end{cvitems}
  }

  %---------------------------------------------------------
%   \cventry
%   {Participant} % Job Title
%   {ISMTF Senior Mathematics Competitions \& International Junior Kangaroo Competitions} % Organisation
%   {Vienna, Austria \& Moscow, Russia} % Location
%   {Mar. 2015, Mar. 2014, 2009, 2008} % Date(s)
%   {
%     \begin{cvitems} % Description(s) of tasks/responsibilities
%       \bsep{} Participated in the competition in a team of international students to further my insight into mathematics.
%       \bsep{} Used the experience to further my communication skills and to   learn from representatives of foreign to me curricula and teaching methods.
%     \end{cvitems}
%   }
\end{cventries}

% \input{sections/honors.tex}
% \input{sections/presentation.tex}
% \input{sections/writing.tex}
% \input{sections/committees.tex}


%-------------------------------------------------------------------------------
\end{document}
