%-------------------------------------------------------------------------------
%	SECTION TITLE
%-------------------------------------------------------------------------------
\cvsection{Projects}


%-------------------------------------------------------------------------------
%	CONTENT
%-------------------------------------------------------------------------------
\begin{cventries}

%---------------------------------------------------------
    \cventry
    {Game Developer/Designer, Git Master, Asset Creation}
    {\href{https://www.youtube.com/watch?v=u3z3P-rHndg}{\uline{Seth's PyraMall}} --- A Competitive Couch Platformer}
    {Zurich, Switzerland}
    {Feb. 2022 -- May. 2022}
    {
        \begin{cvitems}
        \item As a part of the \href{https://gtc.inf.ethz.ch/education/game-programming-laboratory.html}{\uline{Game Technology Center}}'s Game Programming Lab, in a team of six, I developed a 2D retro-style platformer game using C\#, .NET, and the Monogame framework.
        \item Led the team: set priorities, defined  and assigned individual tasks,  mainated theproject's issue log, as well as the projects Git etiquette and history.
        \item As a developer, I developed the character controller, contributed to the sound-effect system and the menu GUI system.
        \item I learned how to apply C\# and .NET during the project. How to apply OOP design-patterns in the language, as well as how to keep a clean and well-documented codebase.
        \item I \href{https://youtu.be/AuFtp9p35H4?t=1211}{presented the final game to an audience of c. 200}; won the \href{https://youtu.be/AuFtp9p35H4?t=5614}{\textbf{\uline{Jury Award}} from the triple-A Studio Gobo} for the game.
        \end{cvitems}
    }

    %---------------------------------------------------------
    \cventry
    {Developer}
    {Tool for Analyzing Projects and Resouce Allocation Trends}
    {Zurich, Switzerland}
    {Sep. 2021 -- Jan. 2022}
    {
        \begin{cvitems}
        \item As a part of the \href{https://analytics-club.org/wordpress/hack4good/}{\uline{Hack4Good}} program, in a team of four, I developed an NLP analysis package for \href{https://www.helvetas.org/en/switzerland}{\uline{Helvetas}} to help the company better understand the distribution of their decentrelized internaltional projects.
        \item I developed and tested NLP models for the analysis, implemented the package's business logic and partly designed its web-based UI. I used Docker to share the prototype with the company.
        \item The package allows the user to analyze a dataset of projects, classify them into areas of development, and identify trends appearing in their descriptions and outcomes to better allocate resources.
        \end{cvitems}
    }

    %---------------------------------------------------------
    % \cventry
    % {Developer}
    % {Classification of Heart Arrhythmia from ECG signals using DL}
    % {Zurich, Switzerland}
    % {Nov. 2021}
    % {
    %     \begin{cvitems}
    %     \item I implemented and adapted the Stanford ECG2 architecture using PyTorch for submission during Advanced Machine Learning at ETH Zurich.
    %     \item Deployed remote GPUs for accelerated training using Docker images.
    %     \item Solution improved team's leaderboard position from c. 140th to 6th place out of c. 200 teams.
    %     \end{cvitems}
    % }
    
    % %---------------------------------------------------------
    % \cventry
    % {Developer}
    % {Analyzing and Augmenting PWC-Net for Multi-Human Optical Flow}
    % {Zurich, Switzerland}
    % {May. 2020}
    % {
    %     \begin{cvitems}
    %     \item In a team of three, I analyzed and carried out augmentations to the PyTorch implementation of PWC-Net for Human Optical Flow as the final project for Final Project in Machine Perception at ETH Zurich.
    %     \item I attempted to improve the network's overfitting resilience via convolutional DropBlock and a revised data augmentation pipeline.
    %     \item I designed and carried out iterative experiments on remote GPUs to train and tune the model.
    %     \end{cvitems}
    % }

    %---------------------------------------------------------
    \cventry
    {Graphic and UI/UX Designer, Frontend Developer} % Job title
    {\href{https://github.com/davzzar/clim-EX}{\uline{Clim-EX}}: The Climate Animation Explorer} % Organization
    {Oxford, UK} % Location
    {Nov. 2019} % Date(s)
    {
        \begin{cvitems}
        \item Selected to participate \href{http://www.ox.ac.uk/students/news/2019-10-14-oxford-hack-2019}{\uline{Oxford University Hackathon 2019}. In a team of four, built a project promoting climate change discussion and exploration.
        \item Designed and built the frontend. Our submission was one of the lead contenders to the ``Hacker's Choice'' award at the hackathon.
        \end{cvitems}
    }

    %---------------------------------------------------------
    \cventry
    {Designer and Programmer} % Job title
    {\href{https://github.com/dtetruash/solved-af}{\uline{Solved-AF}}: Argumentation Framework Solver and ``SAF-link'' API} % Organization
    {London, UK} % Location
    {Oct. 2019 --- Aug. 2020} % Date(s)
    {
        \begin{cvitems}
            \item I designed and developed an \href{https://www.sciencedirect.com/science/article/pii/000437029400041X}{\underl{Argumentation Framework}} solver for non-monotonic reasoning as part of my Bachelor's Thesis with the goal of producing a solver for educational and exploratory purposes.
            \item To aid in the exploratory role, I developed a REST API prototype which enables web clients to use the rich the plethora of existing Argumentation Framework solvers bypassing the knowledge barrier to use. 
            \item The thesis achieved a first-class mark, in part due to the implementation.
        \end{cvitems}
    }

    %---------------------------------------------------------
    % \cventry
    % {Designer and Programmer} % Job title
    % {Collegiate Information Portal (KCL-info)} % Organization
    % {} % Location
    % {Oct. 2019 --- Present} % Date(s)
    % {
    %     \begin{cvitems}
    %         \bsep Currently developing an information retrieval web API for use by staff and students of KCL. The main purpose is to provide easy access to important and up-to-date information around campus and all of its aspects in one unified place.
    %     \end{cvitems}
    % }
\end{cventries}

%---------------------------------------------------------
